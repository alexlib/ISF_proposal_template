\documentclass[a4paper,12pt]{report}
\usepackage{natbib}
\usepackage[utf8]{inputenc}
\usepackage[margin=2cm]{geometry}
\usepackage{setspace}
\usepackage{url}
\usepackage{graphicx}
% \usepackage[colorinlistoftodos]{todonotes}
\usepackage{xcolor}
\usepackage{xfrac}
\usepackage[normalem]{ulem}
\usepackage{tikz}
\usepackage{multicol}
%\usepackage{siunitx}
\usepackage{comment}
\usepackage{diagbox}
\usepackage{subcaption} %to have subfigures available
%\usepackage{cite}
\newcommand{\bfl}{\bf \large}
%\doublespacing
%\usepackage[normalem]{ulem}
\usepackage{mathtools}
\usepackage{amssymb,amsmath}
\usepackage{wrapfig}
\usepackage[small,compact]{titlesec} 
% modify section titles and the spacing above/below them, resulting in space savings.
%\usepackage{titlesec}
\titleformat*{\subsubsection}{\normalfont}
\usepackage{wrapfig}
\linespread{1.5}
\usepackage{times} % use Times font instead of the default, saving significant space.
\usepackage[small,compact]{titlesec} % modify section titles and the spacing 
% personalized comments:
\begin{document}
%\section*{\Large ISF SCIENTIFIC PROPOSAL}
%\section{INVESTIGATORS}
%\subsection*{PRINCIPLE INVESTIGATORS
\noindent Proposal number: XXX/XX Fluid Dynamics\\*
PI Name: Israel Israeli \\*
%\emph{Applied Mathematics, Division of Environmental Sciences, The Israel Institute for Biological Research, Ness Ziona 94100, Israel.}\\*
%{\bfl Alex Liberzon} (PI)\\*
%\emph{Turbulence Structure Laboratory, School of Mechanical Engineering, Tel Aviv University, Tel Aviv 69978, Israel.}\\*
%{\bfl Scientific category:} Engineering or Atmospheric\\*
%{\bf Title: }\\*


%{\bf Keywords:} gentle topography, two-dimension hill, canopy flow and transport, environmental wind tunnel, canopy roughness layer (CRL), double-averaged Navier-Stokes, turbulent scalar transport, Fast Flame Ionization Detector (FFID), Laser Doppler Anemometer (LDA), Particle Image Velocimetry (PIV), Pressure measurements.\\* 
%\subsection*{ABSTRACT AND PROGRAM}
%{\bfl Scientific Abstract}\\*
%{\bfl Yardena Bohbot-Raviv}\\*
%{\bfl Application \#2701-20}\\*
%{\bfl Title: XXX}\\*
% General motivation to study transport over hilly canopies -- fluxnet, airpollution, landscape echology
%The exchange of scalars between vegetation and the atmosphere has direct implications on a large number of important environmental problems such as climate change, air and water quality, urban climate and landscape ecology. For instance, a world-wide project, abbreviated FLUXNET~\citep{Baldocchi2001} provides long-term eddy-covariance flux measurements of carbon dioxide, heat and water vapour across different biomes and climates~\citep[e.g.][]{Brugger2019}  to quantify the CO$_2$ sink for  climate change predictions. However, our understanding of how different biological sources and sinks are related to the turbulent fluxes measured on a single tower in a forest is complex mainly due to the inhomogeneity of the atmospheric landscape. With this respect, we believe that studies of scalar transport within and above canopies under controlled laboratory conditions are required ~\citep{Belcher2012}. In the past few decades a combined experimental (field and laboratory)~\citep{Patton2001,Raupach1981, Poggi2007a,Poggi2007b,Poggi2007c} and theoretical effort~\citep{Raupach1987, Finnigan2000, Katul1997a} has been paid to study canopy flows. This effort, to a large extent in idealized conditions of homogeneous canopy, flat terrain, neutral or slightly unstable conditions~\citep[e.g.][among others]{Kaimal1994}, improved our understanding of processes within and above forest canopies. Since real forests are neither homogeneous nor flat, there is a basic need to study how the topography or inhomogeneity affect the canopy air flow and associated scalar transport~\citep{Belcher2012}. We propose an experimental approach to reveal the basic mechanisms of the scalar transport through and above a model canopy on an idealized complex topography, a so-called ``forested hill''. We will create an idealized model of a two-dimensional (2D) gentle hill  covered with a canopy in the environmental wind tunnel at IIBR. In order to measure both within and above the canopy and be able to connect the canopy flow dynamics with the scalar transport and with the turbulent scalar fluxes in the atmosphere, we will create a dense and a sparse canopy array of thin rectangular plates of height $h_c$~\citep{Moltchanov2011}. Furthermore we plan for a given hill slope ($H/L$) to vary the drag exerted by the dense canopy on the airflow over the hill by changing the canopy leaf area density, $a(z)~[\mathrm{m}^2/\mathrm{m}^3]$ in two ways (i.e., vertical distribution of frontal area density, $LAI=\int_0^{h_c} a(z)$).  The first involves changing the density of elements (i.e., reduce the number of elements per unit area into a 'sparse canopy') and the second involves changing the shape of the leaf area index, while keeping the LAI (i.e., total solid density) constant~\citep{Gromke2018, Efstathiou2018}. These changes in the hill roughness will affect the presence of recirculating zones within the canopy layer, which in turn, will affect the mechanisms under which scalar fluxes distribute within and above the forested hill.The novelty of the proposal is in the connection of the flow within and above the canopy to the turbulent scalar fluxes in the presence of topography and different canopy densities. To obtain this type of unique information   we need to use: a) Laser Doppler anemometer (LDA) measuring point-flow within and above the canopy along the hill, b) particle image velocimetry (PIV) to reveal the two-dimension spatial patterns in the flow field within and above the modeled canopy, c) pressure transducers along the hill surface collecting horizontal streamwise pressure gradient and quantifying the recirculation zones, and more importantly, d) concentration 'point-measurements' of a passive scalar, released from controlled lines sources distributed along the hill, using a two-head fast-flame ionization (FFID) detectors. The unique synchronized FFID-LDA measurement at high sampling rates (above $0.5$ KHz) will provide the turbulent scalar fluxes through the simultaneously measured scalar concentration and two velocity components (streamwise and vertical). The measurements will span the canopy roughness layer (CRL) which includes the canopy layer up to 2-5 times its height. The collected data will help to understand how the spatial  distribution of the turbulent scalar fluxes are affected by coupling of the canopy properties (e.g., resistance drag and leaf area density profile) with the pressure gradients over the gentle hill through the recirculation mechanism. 
\vspace{-2em}



\subsection*{Personnel}

\begin{table}[h!]
    \centering
    \begin{tabular}{|l|l|c|c|c|c|c|}
    \hline
        Name &  Role  & \% time devoted & \multicolumn{4}{|c|}{Salaries [NIS/year]} \\
        \hline
        \multicolumn{3}{|c|}{}& year 1 & year 2 & year 3 & year 4 \\
        \hline
        PI Name 1 & PI & 10 &  0 & 0 & 0 & 0 \\
        PI Name 2 & PI & 10 & 0 & 0 & 0 & 0\\
        \hline
        Student & PhD & 100 & 100,000 & 100,000 & 100,000 & 100,000 \\
        \hline
        Student & PhD & 100 & 100,000 & 100,000 & 100,000 & 100,000 \\
        \hline
        Technical assistance (TAU) & Technician  & 10 &  20,000 & 20,000&  &  \\
        \hline
        Technical assistance (IIBR) & Technician  & 10 & 35,000 & 35,000 &  &  \\
        \hline
        \hline
        Total Personnel & & &255,000 &255,000 &200,000 &200,000\\
        \hline
    \end{tabular}
\end{table}


\newpage
\subsection*{Supplies, Materials \& Services }

\begin{table}[h!]
\centering
\begin{tabular}{|l|c|c|c|c|}
\hline
Item     & \multicolumn{4}{|c|}{Requested sums [NIS/year]} \\
\hline
         & 
year 1 & year 2 & year 3 & year 4 \\
\hline
grade-4 ethane tracer,calibration mixture,& & & & \\
seeding aerosol tracer, smoke & 
10,000 & 10,000 & 5,000 & 5,000 \\
\hline
Dense layer in FH3 --  PPI-10 bed-foam ($\sim 8 m$) & 20,000 &&&\\
\hline
Student travel expenses for scientific conferences (X2) &
4,000 & 4,000 & 8,000 & 8,000 \\
\hline
\hline
Total supplies, materials \& services & 
34,000 & 14,000 & 13,000 & 13,000\\
\hline
\end{tabular}
\end{table}



\subsection*{Computers}

\begin{table}[h!]
\centering
\begin{tabular}{|l|c|c|}
\hline
  Item  & Amount & Total Price [NIS] \\
\hline
Personal desktop computer & 2 & 8,000  \\
\hline 
Laptop computer & 2 & 20,000\\
\hline
\hline
Total  computers & 4 & 28,000\\
\hline
\end{tabular}
\end{table}


\subsection*{Miscellaneous }

\begin{table}[h!]
\centering
\begin{tabular}{|l|c|c|c|c|}
\hline
    & \multicolumn{4}{|c|}{Requested sums [NIS/year]} \\
\hline
 Photocopies and office supplies& 
1,000 & 1,000 & 1,000 & 1,000 \\
\hline
Publication charges in scientific journals &  
0 & 0 & 3500 & 3500 \\
\hline
Professional literature &
0 & 0 & 0 & 0\\
\hline\hline
Total miscellaneous & 
1,000 & 1,000 & 4,500 & 4,500\\
\hline
\end{tabular}
\end{table}



\subsection*{Equipment } 
\begin{table}[h!]
\centering
\begin{tabular}{|l|c|}
\hline
  Item  & Price [NIS] \\
\hline
Two-dimension gentle hill via CNC & 50,000  \\
\hline
Rectangular thin plates forest canopy model ($\sim 8\times 2 m^2$ with $\sim 1000$ plates $m^{-2}$) & 35,000\\
\hline
%\hline
%Pressure scanning system (Scanivalve Corp.) &  60,000  \\
\hline
Total & 80,000\\
\hline
Other expenses (including shipping,
installation, customs and taxes) & 0,000 \\
\hline
\hline
\hline
Total & 70,000 \\
\hline
 Funds requested from ISF & \\
 \hline
\end{tabular}
\end{table}

\newpage

\subsection*{Justification for requested budget }

We have a lot of stuff, but we need more. Good scientific tools are expensive. 

\bibliographystyle{apalike2}
% up to 5 pages
\bibliography{references}
\end{document}
